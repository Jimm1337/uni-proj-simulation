\documentclass[a4paper,12pt]{article}
\usepackage[utf8]{inputenc}
\usepackage[T1]{fontenc}
\usepackage[polish]{babel}
\usepackage{setspace}
\usepackage{times}
\usepackage{fancyhdr}
\usepackage{nameref}

\title{Merchant simulation - projekt symulacji na laboratorium Programowanie Obiektowe.}
\author{Gusta Oskar, Zajęcia: K00-65g (wt 19:05)}
\date{23.06.2022r.}

%\pagestyle{fancy}
%\fancyhf{}
\cfoot{\thepage}

\begin{document}

  \onehalfspacing
  \maketitle
  \newpage
  \tableofcontents
  \newpage

  \section{Koncept}

  Program wykonuje symulację której celem jest pokazanie przebiegu handlu na zadanych algorytmach i z określoną losowością.

  \section{Opis działania}

  Po uruchomieniu projektu z zadanymi argumentami decydującymi o losowości zostaje użytkownikowi przedstawiona możliwość kontynuowania symulacji która wcześniej została zapisana.
  Jeśli użytkownik posiada plik z zapisanym stanem symulacji, może on go wczytać w tym kroku.
  W przeciwnym wypadku wyświetlane jest pytanie o typ strategii i algorytm odpowiedzialny za przemieszczanie się.
  Następnie po wczytaniu lub wybraniu stworzeniu nowej symulacji użytkownikowi jest prezentowana właściwa część programu.
  Na górze ekranu wyświetlane są wioski z podstawowymi danymi, są one losowo generowane co trzecią turę symulacji i na początku.
  Pod wioskami znajdują się dane kupca: pozycja, ilość i konsumpcja jedzenia na jeden 'dzień' symulacji, ilość pieniędzy, informacja czy kupiec został zaatakowany na ostatniej ścieżce do wioski oraz ilość posiadanych dóbr.
  Użytkownik ma dostęp do konsoli sterowania z poziomu której może poruszać symulację do przodu o jeden lub kilka iteracji a także zapisać, wyjść oraz otworzyć pomoc z poziomu symulacji.

  Co każdy epoch kupiec przemieszcza się do wioski za pomocą wybranego algorytmu i płaci cenę proporcjonalną do przebytej drogi.
  W trakcie podróży kupiec może zostać zaatakowany z prawdopodobieństwem które zwiększa się z odległością konieczną do przebycia.
  Jeśli dojdzie do ataku kupiec traci część pieniędzy oraz część z każdego typu posiadanego dobra, w tym jedzenia.
  Następnie sprawdzana jest ilość jedzenia, jeśli jest nie wystarczająca symulacja kończy się i użytkownik jest pytany o zapis, po czym następuje wyjście z programu.
  Jeśli zapas jedzenia jest wystarczający następuje przejście do sekcji sprzedaży i kupna.
  Algorytmy kupują i sprzedają w proporcji z góry określonej, uporządkowanej przez ceny dóbr w wiosce.
  Cykl programu powtarza się do momentu końca symulacji lub manualnego wyjścia z opcjonalnym zapisem.

  Wybrane parametry mają wpływ na ogólną losowość symulacji oraz dotkliwość ataku, jeśli on nastąpi.
  Wybrana strategia decyduje o losowości, konsumpcji jedzenia oraz o obniżce cen kupna.
  Wybrany algorytm przemieszczania się odpowiada za to czy nasz kupiec będzie przemieszczał się do wioski która jest najbliżej czy do tej której \textit{Price Index} obliczany jako średnia algorytmiczna cen jest najniższy.

  \section{Szczegóły techniczne}

  Projekt posiada poprawne zapisywanie i odczytywanie stanu symulacji do formatu JSON z uwzględnieniem polimorfii oraz rekursywnych odwołań (\textit{io.json.Converter}).
  Projekt korzysta z polimorfii na potrzeby wyboru strategii (Wykorzystanie \textit{simulation.environment.Epochs}, Implementacja \textit{simualtion.strategy.StrategyType}).
  Projekt korzysta z klas abstakcyjnych oraz dziedziczenia (Na przykład \textit{simulation.goods.StockBase}).
  Projekt korzysta z Javadoc do tworzenia dokumentacji poszczególnych klas, metod, stanów i pakietów.
  Projekt posiada schemat klas zamieszczony w tym samym folderze w systemie git.
  Projekt korzysta z kontroli źródła.

  \newpage
  Wykonano przy użyciu \LaTeX.

\end{document}
